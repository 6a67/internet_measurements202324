\section{Conclusion}
This study delves into an analysis of \ac{DoS} attacks targeting authoritative nameservers, drawing upon data from the UCSD Network Telescope and OpenINTEL datasets alongside additional measurements.

The frequency of DNS attacks is relatively low, with the majority of attacks failing to disrupt DNS resolution for the vast majority of end users.
However, certain attacks can have a substantial impact on smaller nameservers and pose real-world threats to service availability.

Anycast configurations exhibit greater resilience to attacks compared to unicast configurations.
Diversifying the network infrastructure further by deploying nameservers on multiple subnets can also enhance resilience.
Therefore, DNS operators should consider implementing anycast nameservers and dispersing them across multiple subnets.

The combined datasets offer a unique perspective on DNS attacks, providing insights from both passive and active network monitoring.
The findings reveal that attackers employ a diverse range of ports and protocols to target authoritative DNS servers, with many attacks targeting ports other than port 53, the standard port for DNS.

Overall, the study's findings highlight that DNS attacks pose a genuine threat.
However, by implementing the countermeasures discussed in this paper, DNS operators can effectively mitigate the impact of attacks and strengthen the resilience of their DNS infrastructure.