\section{Ethical Considerations} \label{sec:ethicalconsiderations}
The release of information indicating that specific companies have been subjected to attacks raises ethical concerns.
Companies often hesitate to disclose attack details due to factors like reputational damage and legal liabilities.
However, there are also ethical justifications for disclosing attack information, such as raising awareness of attack vectors and facilitating the development of countermeasures.

In the context of the data collection methods utilized for this study, the ethical concerns are relatively minor.
The UCSD Network Telescope data is gathered through passive network monitoring and does not send any packets to the target systems.
OpenINTEL is an active network monitoring project, but its impact is negligible as it only sends a limited number of queries to each target system \cite{openintel}.
The reactive measurements conducted also have a minimal impact, as they only involve sending a small and evenly distributed number of queries.

Furthermore, IP addresses were removed from the published data, with only the associated companies being disclosed.
Exceptions were made only in instances where the data was already publicly available.

Overall, the ethical concerns stemming from the data collection methods employed in this study are minimal.