\section{Scope of Attacks against Authoritative DNS Servers} \label{sec:scopeofattacksagainstauthoritativednsservers}
The preceding examples highlight the impact of specific attacks targeting authoritative DNS servers.
However, the goal of the study is to infer a more general overview of the scope of these attacks.
The following section delves into the insights that can be extracted from the data and the implications of these findings.
\subsection{Attack Targets}
The UCSD Network Telescope data indicates that approximately 0.5\% to 2\% of all potential attacks are directed at the DNS infrastructure.
While the absolute number of attacks is relatively low, a successful attack on a nameserver could have a significant impact on millions of domains and their users.
On average, each attack potentially affects a small number of domains.
However, the measurements show eight peaks that impacted 10 million domains, or approximately 4\% of the domains covered by OpenINTEL.
These attacks had negligible impacts on resolution time.
Most attacks target large hosting companies such as Google, Unified Layer, and Cloudflare, with very little impact on the overall infrastructure \cite{Sommese2022DDoSDNS}.


\subsection{Attack Characteristics}
The analysis of \ac{DoS} attack characteristics shows that a significant majority (80.7\%) of attacks targeted a single port and protocol. The most common protocol used is TCP (90.4\%), followed by UDP (8.4\%) and ICMP (1.2\%).

While DNS originated as a UDP-based protocol, the use of TCP has increased over the years, and many authoritative nameservers have adopted TCP as a protocol \cite{Mao2022AssessingSF}.

The most common ports targeted by TCP attacks were port 80 (HTTP), port 53 (DNS), and port 443 (HTTPS), accounting for 37.1\%, 30.1\%, and 26.1\% of attacks, respectively.
UDP attacks targeted a more diverse set of ports, with one-third of attacks targeting port 53 (DNS).

The persistent popularity of TCP SYN flooding attacks \cite{cloudflareddosreport2022} \cite{netscoutdirectpath}, even when overtaken by other attack vectors \cite{cloudflareddosreport2023q1} \cite{cloudflareddosreport2023q2}, may be the reason for the high number of TCP-based attacks.
TCP SYN flooding attacks work by sending a large number of SYN packets to a target server, exploiting the TCP three-way handshake, and leaving the target with half-open connections that consume resources \cite[688-690]{Stallings_William2022-06-06}.

Notably, the majority of attacks targeting DNS nameservers did not target port 53, the port used by DNS.

Without knowing the attackers' goals and motivations, it is difficult to infer the reasons for this behavior and the chosen attack vectors.
However, even attacks not targeting port 53 can still have an impact on the DNS infrastructure, as the network and server resources can still be overwhelmed by the attack traffic \cite{Sommese2022DDoSDNS}.


\subsection{Attack Impacts}
Most identified \ac{DoS} attacks have no impact on the ability to resolve domain names, with only 1\% of attacks resulting in a resolution failure.
However, smaller nameservers (10-10,000 domains) are more vulnerable, even if they use an anycast configuration.

Similarly, the majority of attacks have no impact on the resolution time.
In approximately only 5\% of cases, the resolution time increases by more than 10-fold, and in fewer than 2\% of cases, the resolution time increases by more than 100-fold.
These attacks also primarily target smaller nameservers.
Attacks targeting larger nameservers are less effective, but can still lead to a 2- to 3-fold increase in resolution time \cite{Sommese2022DDoSDNS}.

Overall, most \ac{DoS} attacks on DNS authoritative servers are ineffective, but some can have a significant impact.
The real-world impact of these attacks will vary depending on the end user's DNS resolver.
Most DNS resolvers cache query results for a certain amount of time, so an increase in resolution time or failure of resolution will only be noticeable if the cached result expires during the attack.


\subsection{Effect of Intensity and Duration on the Attack Impact}
Determining the severity of an attack from the UCSD Network Telescope data is a complex task.
As noted in \Cref{subsec:limitations}, the data only permits the identification of \ac{RSDoS} attacks.
If an attacker employs alternative attack vectors, the data will only reflect the \ac{RSDoS} component, making it challenging to assess the true intensity.
Furthermore, even when the Telescope indicates a high-intensity attack, determining the impact proves difficult as it hinges on the target's infrastructure.
Overall, while the data enables the identification of potential attacks, it does not facilitate a detailed analysis of all attack vectors or a correlation with the impact \cite{moore2006}.

Most effective attacks were short-lived, ranging from 15 to 60 minutes.
Given that the Telescope only detects \ac{RSDoS} attacks, a successful attack may appear as a short-lived event, even if it persisted for an extended period, as backscatter traffic may be hindered if the victim cannot respond to the incoming traffic.

In summary, the UCSD Network Telescope data is valuable for identifying potential attacks, but it falls short of providing a comprehensive assessment of their impact \cite{Sommese2022DDoSDNS}.


\subsection{Effect of Anycast Deployment and Network Diversification on the Attack Impact} \label{subsec:soa_anycastdeploymentandnetworkdiversification}
Consistent with the findings in \Cref{sec:observableattacksandtheirimplications}, the broader data indicates that anycast deployments exhibit greater resilience, experiencing only a 1- to 1.5-fold increase in RTT while under attack.
Partial anycast deployments, where only a portion of nameservers employ anycast, demonstrated reduced resilience compared to anycast-only deployments.
Unicast-only deployments proved to be the most susceptible to attacks.

Furthermore, the data suggests that dispersing nameservers across multiple subnets enhances resilience.
Nameservers residing on the same \texttt{/24} subnet likely share the same network link.
Sixty percent of the nameserver sets that experienced outages were hosted on the same \texttt{/24} subnet.
Distributing nameservers across two subnets improves resilience, as only 30\% of nameserver sets that experienced outages spanned two \texttt{/24} subnets.
Spreading nameservers across three subnets further strengthens resilience, with nameserver sets encompassing three or more subnets accounting for only 10\% of those that experienced outages \cite{Sommese2022DDoSDNS}.