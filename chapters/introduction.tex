\section{Introduction}
The \acf{DNS} is a fundamental component of the Internet, translating domain names into numerical IP addresses, enabling users to access websites and a vast array of online resources.
Its role is indispensable to the modern Internet, and a successful \ac{DoS} attack targeting \ac{DNS} nameservers can have a catastrophic impact on global connectivity.
Without \ac{DNS}, navigating the web, sending emails, or utilizing numerous online services would become impossible.

\ac{DoS} attacks pose a significant threat to the \ac{DNS} infrastructure \cite{dynanalysis} \cite{wireddyn}.
These attacks involve overwhelming \ac{DNS} servers with a flood of traffic, rendering them unavailable to legitimate users.
\ac{DNS} \ac{DoS} attacks are particularly effective due to the high volume of traffic typically handled by \ac{DNS} servers.

The successful execution of a \ac{DNS} \ac{DoS} attack can have far-reaching consequences for the Internet.
It can trigger widespread outages of websites, applications, and other online services.
Moreover, \ac{DNS} \ac{DoS} attacks can be employed to target specific organizations or industries, causing targeted disruptions.

The threat of \ac{DNS} \ac{DoS} attacks underscores the critical need for \ac{DNS} reliability.

This paper will discuss a study that analyzed the impact of DNS \ac{DoS} attacks on authoritative nameservers.
First, the fundamentals of \ac{DNS} servers are laid out.
Subsequently, the methodology employed and specific examples of \ac{DNS} \ac{DoS} attacks are presented, followed by a broader analysis of \ac{DNS} \ac{DoS} attacks.
Finally, the paper explores the implications for \ac{DNS} and domain operators, along with potential countermeasures.