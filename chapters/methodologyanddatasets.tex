\section{Methodology and Datasets} \label{sec:methodologyanddatasets}
To identify and analyze \ac{DoS} attacks against authoritative DNS servers, an analysis was conducted on a combined dataset from the UCSD Network Telescope and OpenINTEL measurement project.
This analysis spanned a 17-month period, encompassing data from November 1, 2020, to March 31, 2022.


\subsection{UCSD Network Telescope - Passive Network Monitoring} \label{subsec:ucsd}
A \ac{RSDoS} attack utilizes randomly spoofed source IP addresses to obscure the attack's origin and hinder the victim's ability to filter out malicious traffic.
Unable to distinguish between legitimate and spoofed traffic, the victim attempts to respond to all incoming requests.
These outgoing packets, generated in response to spoofed requests, are known as \textit{backscatter} traffic \cite{caidarsdos}.

The UCSD Network Telescope, a network comprising a large number of unused IP addresses (roughly 1/256th of all IPv4 addresses), passively collects traffic directed to these spaces.
Since these IP addresses are unassigned, any traffic received is considered suspicious, enabling the capture of backscatter traffic.
By analyzing this backscatter data, DoS attacks can be identified and insights into their scale, target, and methodology can be obtained \cite{caidatelescope} \cite{10.1145/1132026.1132027}.


\subsection{OpenINTEL - Active DNS Measurements} \label{subsec:openintel}
The OpenINTEL measurement project conducts daily DNS queries on approximately 70\% of all registered domain names worldwide (as of Oct.\ 2023) \cite{openintel} \cite{dnib2023q2}.
These measurements encompass resolution time, enabling the detection of anomalies in DNS latency and reachability.
Since OpenINTEL does not provide information about the authoritative DNS server that answered the query, the IP addresses of nameservers associated with each domain are aggregated into a set.

Over a 5-minute interval, the number of domains resolved by OpenINTEL, the average, minimum, and maximum \acf{RTT}, and the number of errors are collected for each set by aggregating data across all domains within the same set.
Identifying an impact on the nameservers is achieved by comparing the average RTT of the set to the average RTT of the same set on the previous day.
This approach allows for a meaningful comparison within the same set and allows for swift adaptation if the DNS infrastructure, and consequently, the latency, undergoes changes \cite{Sommese2022DDoSDNS}.


\subsection{Methodology}
The UCSD Network Telescope data can be leveraged to identify potential \ac{RSDoS} attacks against IP addresses.
By cross-referencing the IP addresses of nameservers with elevated \ac{RTT} detected by OpenINTEL and the IP addresses identified by the UCSD Network Telescope data, potential \ac{RSDoS} attacks targeting authoritative DNS servers can be pinpointed.

Utilizing the aggregated data from the OpenINTEL dataset enables the extraction of domain names served by the nameservers identified as potential targets.
Measuring the RTT of these domains allows for an analysis of the impact of these attacks \cite{Sommese2022DDoSDNS}.


\subsection{Limitations} \label{subsec:limitations}
Gathering and identifying \acf{DoS} attacks is a complex undertaking.
While combining data from the UCSD Network Telescope and OpenINTEL facilitates the detection of potential DoS attacks, it falls short of providing a comprehensive analysis.

Specifically, the UCSD Network Telescope data only permits the identification of RSDoS attacks against IP addresses and is restricted to IPv4, excluding IPv6 addresses.
Translating identified RSDoS attacks against IPv4 addresses to their IPv6 counterparts may be feasible, as IPv4 and IPv6 services often share network and server infrastructure \cite{Beverly2015ServerSI}.

One limitation of OpenINTEL's random nameserver selection approach is its inability to identify attacks targeting specific nameservers.
However, this approach provides a measurement of the overall end-user experience of resolving a domain name, as a typical end user does not select a specific nameserver.
Additionally, due to the inherent nature of the anycast infrastructure, it is possible that ongoing attacks in specific geographic regions may not be detected using the OpenINTEL dataset.

To address these limitations, an additional reactive measurement was implemented in conjunction with the dataset combination for attacks occurring after January 2022.
Upon detecting a potential DoS attack, all nameservers belonging to a potentially attacked domain are queried iteratively.
For each attack, 50 related domain names are queried evenly spread every 5 minutes during the attack and the following 24 hours.
While this approach also has limitations, originating from a single network in the Netherlands, it does provide a more detailed analysis of the attack's impact \cite{Sommese2022DDoSDNS}.
