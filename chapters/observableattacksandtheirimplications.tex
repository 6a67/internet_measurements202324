\section{Exemplary Attacks and their Implications} \label{sec:observableattacksandtheirimplications}
The analyzed data reveals the impact of multiple attacks against different nameservers.
The following section will delve into publicly acknowledged or reported attacks and their repercussions.


\subsection{Attacks against TransIP's Nameservers} \label{subsec:attacksagainsttransipsnameservers}
TransIP, a European domain and web hosting company, experienced two attacks that impaired the DNS resolution performance of its nameservers: one in December 2020 and the other in March 2021.
Both attacks were acknowledged by TransIP \cite{transipddosdecember} \cite{transipddosmarch}.
TransIP utilized three nameservers hosted in a unicast deployment.

The December 2020 attack appears to have primarily targeted only one of the three nameservers, yet OpenINTEL recorded a 10-fold increase in resolution time due to the randomized querying of nameservers.

The March 2021 attack was more severe and targeted all three nameservers.
While the nameservers' resolution time was also negatively impacted, the attack also caused approximately one-fifth of queries to time out, amplifying the attack's noticeable impact \cite{Sommese2022DDoSDNS}.


\subsection{Attacks against Russian Nameservers} \label{subsec:attacksagainstrussiannameservers}
The TransIP example illustrates an attack targeting a commercial entity, while attacks against Russian assets are suggested to be politically motivated.
In March 2022, an attack targeted \texttt{mil.ru}, the domain of the Russian Ministry of Defense \cite{wiredrussiaddos}.
The domain was managed by three unicast nameservers on the same \texttt{/24} subnet.
During the attack, both OpenINTEL and the reactive measurements failed to resolve the domain, and it was reported that the domain was geo-fenced, allowing only requests from within Russia.

In the same month, another attack targeting the nameservers of the state-owned Russian Railway company was observed.
The three nameservers were hosted on two different \texttt{/24} subnets and used unicast.
All three servers were targeted by the attack, and degradations in resolution time and timeouts were measured \cite{Sommese2022DDoSDNS}.


\subsection{Implications} \label{subsec:ex_implications}
In an attempt to filter out malicious traffic, TransIP implemented a traffic scrubbing technique.
However, despite this measure, the attacks still had a significant impact on their servers and a substantial portion of their customers.
Their servers were deployed in a unicast configuration across three different subnets in two distinct geographical locations.
This deployment renders the infrastructure heavily reliant on these three physical servers and their corresponding network links.

Similarly, the \texttt{mil.ru} nameservers were hosted on the same subnet, implying that they shared the same network link.
In such configurations, targeting the network links of the servers can be sufficient to execute a successful DoS attack.

The Russian Railway nameservers were hosted on two different subnets, but all three nameservers employed unicast and were individually targeted by the attack.
In a unicast configuration, overwhelming the physical server can lead to a successful attack.

A more diverse infrastructure with more physical servers using anycast and distributed across different networks would allow for the distribution of the load and reduce the point of failure \cite{Sommese2022DDoSDNS}.